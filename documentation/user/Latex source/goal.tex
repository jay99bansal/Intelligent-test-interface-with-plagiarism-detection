\subsection{Motivation}

Plagiarism is the practice of taking credit for someone else's words or ideas. It's an act of intellectual dishonesty. In colleges and universities, it violates honor codes. Presenting others' work without adequate acknowledgement of its source, as though it were one’s own is
Plagiarism and it is clearly a form of fraud. \\
Some forms of plagiarism are obvious. Copying someone else's essay word for word and submitting it as your own but more complex forms of plagiarism exist which might be difficult to capture by simple algorithms. \\
Some forms of Plagiarism are :
\begin{enumerate}
    \item \textbf{Direct plagiarism} is the act of copying another person's work word for word. Inserting a paragraph from a book or article into your essay without including attribution.
    \item \textbf{Paraphrased plagiarism} involves making a few (often cosmetic) changes to someone else’s work, then passing it off as your own.
    \item \textbf{Mosaic plagiarism} is a combination of direct and paraphrased plagiarism. This type involves tossing various words, phrases, and sentences (some word for word, some paraphrased) into your essay without providing attributions.
    \item \textbf{Accidental plagiarism} occurs when citations are missing, sources are cited incorrectly, or an author shares an idea without a citation that isn't as common of knowledge as they thought. Accidental plagiarism is often the result of a disorganized research process and a last-minute time crunch. \cite{plag}
\end{enumerate}
Plagiarism detection is the process of locating instances of plagiarism within a work or document. \cite{wiki}Detection of plagiarism can be undertaken in a variety of ways. Human detection is the most traditional form of identifying plagiarism from written work. This can be a lengthy and time-consuming task for the reader and can also result in inconsistencies in how plagiarism is identified within an organization. So the better way is to use automatically plagiarism detectors to save time and efforts.  

\subsection{Objective}
Goal is to build an efficient algorithm which performs optimal plagiarism detection task on the given set of test responses and provide us with a fair index of plagiarism corresponding to each response. And to build an interface which let Test taker to set tests and student to access the live tests and submit responses.


